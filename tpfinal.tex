\documentclass[conference]{IEEEtran}
\IEEEoverridecommandlockouts
% The preceding line is only needed to identify funding in the first footnote. If that is unneeded, please comment it out.
\usepackage{cite}
\usepackage{amsmath,amssymb,amsfonts}
\usepackage{algorithmic}
\usepackage{graphicx}
\usepackage{textcomp}
\usepackage{xcolor}
\def\BibTeX{{\rm B\kern-.05em{\sc i\kern-.025em b}\kern-.08em
    T\kern-.1667em\lower.7ex\hbox{E}\kern-.125emX}}
\begin{document}

\title{Résilience d'une Architecture Command and Control sur AWS
}

\author{\IEEEauthorblockN{Christophe St-Georges}
\IEEEauthorblockA{\textit{Département de génie informatique et logiciel} \\
\textit{Polytechnique Montréal}\\
Montréal, Canada \\
christophe.st-georges@polymtl.ca}
\and
\IEEEauthorblockN{Jimmy Bell}
\IEEEauthorblockA{\textit{Département de génie informatique et logiciel} \\
\textit{Polytechnique Montréal}\\
Montréal, Canada \\
jimmy.bell@polymtl.ca}
\and
\IEEEauthorblockN{Cédrick Gontran Nicolas}
\IEEEauthorblockA{\textit{Département de génie informatique et logiciel} \\
\textit{Polytechnique Montréal}\\
Montréal, Canada \\
cedrick.nicolas@polymtl.ca}
\and
\IEEEauthorblockN{Celina Ghoraieb-Munoz}
\IEEEauthorblockA{\textit{Département de génie informatique et logiciel} \\
\textit{Polytechnique Montréal}\\
Montréal, Canada \\
celina.ghoraieb-munoz@polymtl.ca}
}


\maketitle

\begin{abstract}
Depuis la nuit des temps, le cloud.Depuis la nuit des temps, le cloud.Depuis la nuit des temps, le cloud.Depuis la nuit des temps, le cloud.Depuis la nuit des temps, le cloud.Depuis la nuit des temps, le cloud.Depuis la nuit des temps, le cloud.Depuis la nuit des temps, le cloud.Depuis la nuit des temps, le cloud.Depuis la nuit des temps, le cloud.Depuis la nuit des temps, le cloud.Depuis la nuit des temps, le cloud.Depuis la nuit des temps, le cloud.Depuis la nuit des temps, le cloud.Depuis la nuit des temps, le cloud.
\end{abstract}

\begin{IEEEkeywords}
nuage,
\end{IEEEkeywords}

\section{Introduction}

\subsection{Contexte}
Bonjour
\subsection{Énoncé du problème}

\subsection{Planification de mesures de sécurité infonuagique}

\subsection{Portée de l'étude}

\section{État de l'art}

\section{Méthodologie}

\subsection{Services infonuagiques et composants}

\subsection{Configuration de l'infrastructure}

\subsection{Surveillance et journalisation}

\subsection{Détection des anomalies}

\section{Expérimentation}

Défi: Azure Student subscription ne permet pas Terraform -> AWS

\subsection{Test de pénétration de l'infrastructure}

\subsection{Simulation de brèche de données}

\subsection{Test de la détection des anomalies}
\subsubsection{Modèle d'apprentissage machine pour la détection d'anomalies}
\subsection{Évaluation de l'efficacité de la surveillance et de la journalisation}

\section{Réponse aux incidents}

\subsection{Planification de la réponse aux incidents (IRP)}

\subsection{Dectection et déclaration d'incident}
\subsection{Analyse de l'incident}
\subsubsection{Analyse des logs}
\subsubsection{Analyse forensique}
\subsubsection{Analyse de la cause à la racine (RCA)}
\subsubsection{RCA vs. Analyse forensique en sécurité infonuagique}
\subsection{Confinement, éradication et récupération}

\subsection{Post-Mortem}
\subsubsection{Communication}
\subsubsection{Durcissement post-incident}
\subsubsection{Défis}

\section{Discussion}

\subsection{Trouvailles clés}

\subsection{Limitations}
redondance pas faite (implique dupliquer contenu terraform)
\subsection{Déductions}

\section{Conclusion et travaux futurs}

\subsection{Retour}

\subsection{Recommandations}

\subsection{Directions futures}

\section{Crédits et remerciements}

\begin{thebibliography}{00}
\bibitem{b1} G. Eason, B. Noble, and I. N. Sneddon, ``On certain integrals of Lipschitz-Hankel type involving products of Bessel functions,'' Phil. Trans. Roy. Soc. London, vol. A247, pp. 529--551, April 1955.
\bibitem{b2} J. Clerk Maxwell, A Treatise on Electricity and Magnetism, 3rd ed., vol. 2. Oxford: Clarendon, 1892, pp.68--73.
\bibitem{b3} I. S. Jacobs and C. P. Bean, ``Fine particles, thin films and exchange anisotropy,'' in Magnetism, vol. III, G. T. Rado and H. Suhl, Eds. New York: Academic, 1963, pp. 271--350.
\bibitem{b4} K. Elissa, ``Title of paper if known,'' unpublished.
\bibitem{b5} R. Nicole, ``Title of paper with only first word capitalized,'' J. Name Stand. Abbrev., in press.
\bibitem{b6} Y. Yorozu, M. Hirano, K. Oka, and Y. Tagawa, ``Electron spectroscopy studies on magneto-optical media and plastic substrate interface,'' IEEE Transl. J. Magn. Japan, vol. 2, pp. 740--741, August 1987 [Digests 9th Annual Conf. Magnetics Japan, p. 301, 1982].
\bibitem{b7} M. Young, The Technical Writer's Handbook. Mill Valley, CA: University Science, 1989.
\end{thebibliography}

\end{document}